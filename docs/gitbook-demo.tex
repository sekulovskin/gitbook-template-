% Options for packages loaded elsewhere
\PassOptionsToPackage{unicode}{hyperref}
\PassOptionsToPackage{hyphens}{url}
%
\documentclass[
]{book}
\usepackage{amsmath,amssymb}
\usepackage{lmodern}
\usepackage{ifxetex,ifluatex}
\ifnum 0\ifxetex 1\fi\ifluatex 1\fi=0 % if pdftex
  \usepackage[T1]{fontenc}
  \usepackage[utf8]{inputenc}
  \usepackage{textcomp} % provide euro and other symbols
\else % if luatex or xetex
  \usepackage{unicode-math}
  \defaultfontfeatures{Scale=MatchLowercase}
  \defaultfontfeatures[\rmfamily]{Ligatures=TeX,Scale=1}
\fi
% Use upquote if available, for straight quotes in verbatim environments
\IfFileExists{upquote.sty}{\usepackage{upquote}}{}
\IfFileExists{microtype.sty}{% use microtype if available
  \usepackage[]{microtype}
  \UseMicrotypeSet[protrusion]{basicmath} % disable protrusion for tt fonts
}{}
\makeatletter
\@ifundefined{KOMAClassName}{% if non-KOMA class
  \IfFileExists{parskip.sty}{%
    \usepackage{parskip}
  }{% else
    \setlength{\parindent}{0pt}
    \setlength{\parskip}{6pt plus 2pt minus 1pt}}
}{% if KOMA class
  \KOMAoptions{parskip=half}}
\makeatother
\usepackage{xcolor}
\IfFileExists{xurl.sty}{\usepackage{xurl}}{} % add URL line breaks if available
\IfFileExists{bookmark.sty}{\usepackage{bookmark}}{\usepackage{hyperref}}
\hypersetup{
  pdftitle={A GitBook Example for Teaching},
  pdfauthor={Nikola Sekulovski},
  hidelinks,
  pdfcreator={LaTeX via pandoc}}
\urlstyle{same} % disable monospaced font for URLs
\usepackage{longtable,booktabs,array}
\usepackage{calc} % for calculating minipage widths
% Correct order of tables after \paragraph or \subparagraph
\usepackage{etoolbox}
\makeatletter
\patchcmd\longtable{\par}{\if@noskipsec\mbox{}\fi\par}{}{}
\makeatother
% Allow footnotes in longtable head/foot
\IfFileExists{footnotehyper.sty}{\usepackage{footnotehyper}}{\usepackage{footnote}}
\makesavenoteenv{longtable}
\usepackage{graphicx}
\makeatletter
\def\maxwidth{\ifdim\Gin@nat@width>\linewidth\linewidth\else\Gin@nat@width\fi}
\def\maxheight{\ifdim\Gin@nat@height>\textheight\textheight\else\Gin@nat@height\fi}
\makeatother
% Scale images if necessary, so that they will not overflow the page
% margins by default, and it is still possible to overwrite the defaults
% using explicit options in \includegraphics[width, height, ...]{}
\setkeys{Gin}{width=\maxwidth,height=\maxheight,keepaspectratio}
% Set default figure placement to htbp
\makeatletter
\def\fps@figure{htbp}
\makeatother
\setlength{\emergencystretch}{3em} % prevent overfull lines
\providecommand{\tightlist}{%
  \setlength{\itemsep}{0pt}\setlength{\parskip}{0pt}}
\setcounter{secnumdepth}{5}
\usepackage{booktabs}
\usepackage{amsthm}
\makeatletter
\def\thm@space@setup{%
  \thm@preskip=8pt plus 2pt minus 4pt
  \thm@postskip=\thm@preskip
}
\makeatother
\ifluatex
  \usepackage{selnolig}  % disable illegal ligatures
\fi
\usepackage[]{natbib}
\bibliographystyle{apalike}

\title{A GitBook Example for Teaching}
\author{Nikola Sekulovski}
\date{2021-09-17}

\begin{document}
\maketitle

{
\setcounter{tocdepth}{1}
\tableofcontents
}
This is a test GitBook based on \href{https://cjvanlissa.github.io/gitbook-demo/}{A GitBook Example for Teaching}.

\hypertarget{prerequisites}{%
\chapter{Prerequisites}\label{prerequisites}}

\hypertarget{getgitbook}{%
\chapter{Get your GitBook}\label{getgitbook}}

\emph{Nikola:} THIS IS A DELIBERATE CHANGE

\hypertarget{editing-the-book}{%
\chapter{Editing the book}\label{editing-the-book}}

\hypertarget{figtab}{%
\chapter{Figures and tables}\label{figtab}}

\hypertarget{examples}{%
\chapter{Examples}\label{examples}}

\hypertarget{open-educational-resources}{%
\chapter{Open Educational Resources}\label{open-educational-resources}}

UNESCO defines Open Educational Resources as \href{https://en.unesco.org/themes/building-knowledge-societies/oer}{\emph{teaching, learning and research materials in any medium -- digital or otherwise -- that reside in the public domain or have been released under an open license that permits no-cost access, use, adaptation and redistribution by others with no or limited restrictions.}}

Open Educational resources can help lighten the workload on individual teachers, who can collaborate with the development of high-quality open access resources, instead of having to develop their own proprietary materials from scratch. Moreover, Open Educational resources are inclusive, lowering the barrier to knowledge acquisition for learners around the world, and enabling lifelong learning for those outside academia.

Many universities support Open Educational Resources. Here are just a few (feel free to \href{https://help.github.com/en/github/collaborating-with-issues-and-pull-requests/creating-a-pull-request}{send a pull request} with other relevant resources).

\begin{itemize}
\tightlist
\item
  \href{https://www.oercommons.org/}{\textbf{OER Commons}}: A freely accessible online library of open educational resources.
\item
  \href{https://uu.figshare.com/}{\textbf{Utrecht University Figshare}}: Open learning objects from Utrecht University.
\item
  \href{https://ocw.jhsph.edu/}{\textbf{Johns Hopkins University OCW}}: Open public health courses and materials.
\item
  \href{https://pitt.libguides.com/openeducation/biglist}{\textbf{University of Pittsburgh OER}}: Big List of Open Educational Resources.
\item
  \href{https://www.merlot.org/merlot/}{\textbf{MERLOT}}: Online learning and support materials and content creation tools, led by an international community of educators, learners and researchers.
\end{itemize}

\hypertarget{compatibility-with-existing-systems}{%
\chapter{Compatibility with existing systems}\label{compatibility-with-existing-systems}}

Many universities offer digital platforms for learning. You might wish to embed your GitBook within these existing systems. Here are two ways in which you might do that. Currently, this section only discusses BlackBoard, but the same principles should apply to other platforms.

\hypertarget{add-a-hyperlink}{%
\section{Add a hyperlink}\label{add-a-hyperlink}}

You can add a link to your GitBook in the BlackBoard course menu by following \href{https://help.blackboard.com/Learn/Instructor/Course_Content/Create_Content/Create_Course_Materials/Link_to_Websites}{this tutorial}.

\hypertarget{embed-the-whole-book}{%
\section{Embed the whole book}\label{embed-the-whole-book}}

You can add a Blank Page to your BlackBoard course menu, and fill that page with a full-size ``iframe'' - a web page within the web page. \href{https://mycampus.maine.edu/web/uc-faculty-portal/education-technology/-/asset_publisher/vEKuFJYvDY5K/content/inserting-an-iframe-into-blackboard?inheritRedirect=false}{This tutorial} explains how to do it. It is possible that your university is blocking this feature, however.

\hypertarget{license-your-gitbook}{%
\chapter{License your GitBook}\label{license-your-gitbook}}

In the spirit of Open Science, it is good to think about making your course materials Open Source. That means that other people can use them. In principle, if you publish materials online without license information, you hold the copyright to those materials. If you want them to be Open Source, you must include a license. It is not always obvious what license to choose.

The Creative Commons licenses are typically suitable for course materials. This GitBook, for example, is licensed under CC-BY 4.0. That means you can use and remix it as you like, but you must credit the original source.

If your project is more focused on software or source code, consider using the \href{https://www.gnu.org/licenses/gpl-3.0.en.html}{GNU GPL v3 license} instead.

You can find \href{https://creativecommons.org/share-your-work/licensing-examples}{more information about the Creative Commons Licenses here}. Specific licenses that might be useful are:

\begin{itemize}
\tightlist
\item
  \href{https://creativecommons.org/share-your-work/public-domain/cc0/}{CC0 (``No Rights Reserved'')}, everybody can do what they want with your work.
\item
  \href{https://creativecommons.org/licenses/by/4.0/}{CC-BY 4.0 (``Attribution'')}, everybody can do what they want with your work, but they must credit you. Note that this license may not be suitable for software or source code!
\end{itemize}

For compatibility between CC and GNU licenses, see \href{https://creativecommons.org/faq/\#Can_I_apply_a_Creative_Commons_license_to_software.3F}{this FAQ}.

  \bibliography{book.bib,packages.bib}

\end{document}
